We use the California Proposition 99 data from \textcite{abadie2010synthetic}, which is a panel data set that contains information on smoking rates and tobacco control policies in California.\footnote{The
data is available at \url{"https://drive.google.com/uc?id=1wD8h8pjCLDy1RbuPDZBSa3zH45TZL7ha&export=download"}.}
 - Table \ref{tab:summary_statistics} presents the summary statistics of the data, which includes the number of observations, mean, standard deviation, minimum, and maximum values for each variable.

\begin{table}[htbp]
    \centering
    \begin{tabular}{lrrrrrrrr}
\toprule
 & count & mean & std & min & 25\% & 50\% & 75\% & max \\
\midrule
state & 1209.00 & 20.00 & 11.26 & 1.00 & 10.00 & 20.00 & 30.00 & 39.00 \\
year & 1209.00 & 1985.00 & 8.95 & 1970.00 & 1977.00 & 1985.00 & 1993.00 & 2000.00 \\
cigsale & 1209.00 & 118.89 & 32.77 & 40.70 & 100.90 & 116.30 & 130.50 & 296.20 \\
lnincome & 1014.00 & 9.86 & 0.17 & 9.40 & 9.74 & 9.86 & 9.97 & 10.49 \\
beer & 546.00 & 23.43 & 4.22 & 2.50 & 20.90 & 23.30 & 25.10 & 40.40 \\
age15to24 & 819.00 & 0.18 & 0.02 & 0.13 & 0.17 & 0.18 & 0.19 & 0.20 \\
retprice & 1209.00 & 108.34 & 64.38 & 27.30 & 50.00 & 95.50 & 158.40 & 351.20 \\
\bottomrule
\end{tabular}

    \caption{Summary Statistics of the California Proposition 99 Data}
    \label{tab:summary_statistics}
\end{table}


The goal of \textcite{abadie2010synthetic} was to estimate the effect of increased cigarette taxes on cigarette sales in California. The panel contains data from 39 states (including California) from 1970 through 2000.
 California passed Proposition 99 increasing cigarette taxes from 1989 onwards. Thus, we have 1 state (California) as the treatment group ( with increased cigarette taxes) and 38 states that were not exposed to the treatment; 19 pre-treatment periods (1970-1988) and 12 post-treatment periods (1989-2000).

 